%\documentclass[12pt]{article}
\documentclass[12pt]{scrartcl}
\title{ELEC 340 Assignment 3}
\nonstopmode
%\usepackage[utf-8]{inputenc}
\usepackage{graphicx} % Required for including pictures
\usepackage[figurename=Figure]{caption}
\usepackage{float}    % For tables and other floats
\usepackage{verbatim} % For comments and other
\usepackage{amsmath}  % For math
\usepackage{amssymb}  % For more math
\usepackage{fullpage} % Set margins and place page numbers at bottom center
\usepackage{paralist} % paragraph spacing
\usepackage{listings} % For source code
\usepackage{subfig}   % For subfigures
%\usepackage{physics}  % for simplified dv, and 
\usepackage{enumitem} % useful for itemization
\usepackage{siunitx}  % standardization of si units

\usepackage[bitstream-charter]{mathdesign}
\usepackage[T1]{fontenc}

\usepackage{tikz,bm} % Useful for drawing plots
%\usepackage{tikz-3dplot}
\usepackage{circuitikz}
\usepackage[ruled,vlined]{algorithm2e}

%%% Colours used in field vectors and propagation direction
\definecolor{mycolor}{rgb}{1,0.2,0.3}
\definecolor{brightgreen}{rgb}{0.4, 1.0, 0.0}
\definecolor{britishracinggreen}{rgb}{0.0, 0.26, 0.15}
\definecolor{cadmiumgreen}{rgb}{0.0, 0.42, 0.24}
\definecolor{ceruleanblue}{rgb}{0.16, 0.32, 0.75}
\definecolor{darkelectricblue}{rgb}{0.33, 0.41, 0.47}
\definecolor{darkpowderblue}{rgb}{0.0, 0.2, 0.6}
\definecolor{darktangerine}{rgb}{1.0, 0.66, 0.07}
\definecolor{emerald}{rgb}{0.31, 0.78, 0.47}
\definecolor{palatinatepurple}{rgb}{0.41, 0.16, 0.38}
\definecolor{pastelviolet}{rgb}{0.8, 0.6, 0.79}
\begin{document}

\begin{center}
	\hrule
	\vspace{.4cm}
	{\textbf { \large \scshape{ Práctica 1 - Regresión lineal y descenso según el gradiente}}}
\end{center}
{\ Javier Sáez \hspace{\fill} Aprendizaje Automático  \\
	\hrule

\paragraph*{Problem 1} %\hfill \newline



\section{Ejercicio 1}

En este ejercicio, se implementará el algoritmo de descenso de gradiente y se aplicará
este sobre varias funciones con el objetivo de estudiar cómo afecta tanto el punto inicial 
como la tasa de aprendizaje $\eta$ a la solución que este algoritmo encuentra.

Lo primero que debemos hacer es recordar en qué consiste el algoritmo de descenso de gradiente, veámoslo en pseudocódigo:


\begin{algorithm}[H]
\SetAlgoLined
\KwResult{Mínimo local de una función}
 parametros: \;
 \While{}{
  instructions\;
  \eIf{condition}{
   instructions1\;
   instructions2\;
   }{
   instructions3\;
  }
 }
 \caption{Descenso de gradiente}
\end{algorithm}
\end{document}
